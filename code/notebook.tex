
% Default to the notebook output style

    


% Inherit from the specified cell style.




    
\documentclass[11pt]{article}

    
    
    \usepackage[T1]{fontenc}
    % Nicer default font (+ math font) than Computer Modern for most use cases
    \usepackage{mathpazo}

    % Basic figure setup, for now with no caption control since it's done
    % automatically by Pandoc (which extracts ![](path) syntax from Markdown).
    \usepackage{graphicx}
    % We will generate all images so they have a width \maxwidth. This means
    % that they will get their normal width if they fit onto the page, but
    % are scaled down if they would overflow the margins.
    \makeatletter
    \def\maxwidth{\ifdim\Gin@nat@width>\linewidth\linewidth
    \else\Gin@nat@width\fi}
    \makeatother
    \let\Oldincludegraphics\includegraphics
    % Set max figure width to be 80% of text width, for now hardcoded.
    \renewcommand{\includegraphics}[1]{\Oldincludegraphics[width=.8\maxwidth]{#1}}
    % Ensure that by default, figures have no caption (until we provide a
    % proper Figure object with a Caption API and a way to capture that
    % in the conversion process - todo).
    \usepackage{caption}
    \DeclareCaptionLabelFormat{nolabel}{}
    \captionsetup{labelformat=nolabel}

    \usepackage{adjustbox} % Used to constrain images to a maximum size 
    \usepackage{xcolor} % Allow colors to be defined
    \usepackage{enumerate} % Needed for markdown enumerations to work
    \usepackage{geometry} % Used to adjust the document margins
    \usepackage{amsmath} % Equations
    \usepackage{amssymb} % Equations
    \usepackage{textcomp} % defines textquotesingle
    % Hack from http://tex.stackexchange.com/a/47451/13684:
    \AtBeginDocument{%
        \def\PYZsq{\textquotesingle}% Upright quotes in Pygmentized code
    }
    \usepackage{upquote} % Upright quotes for verbatim code
    \usepackage{eurosym} % defines \euro
    \usepackage[mathletters]{ucs} % Extended unicode (utf-8) support
    \usepackage[utf8x]{inputenc} % Allow utf-8 characters in the tex document
    \usepackage{fancyvrb} % verbatim replacement that allows latex
    \usepackage{grffile} % extends the file name processing of package graphics 
                         % to support a larger range 
    % The hyperref package gives us a pdf with properly built
    % internal navigation ('pdf bookmarks' for the table of contents,
    % internal cross-reference links, web links for URLs, etc.)
    \usepackage{hyperref}
    \usepackage{longtable} % longtable support required by pandoc >1.10
    \usepackage{booktabs}  % table support for pandoc > 1.12.2
    \usepackage[inline]{enumitem} % IRkernel/repr support (it uses the enumerate* environment)
    \usepackage[normalem]{ulem} % ulem is needed to support strikethroughs (\sout)
                                % normalem makes italics be italics, not underlines
    

    
    
    % Colors for the hyperref package
    \definecolor{urlcolor}{rgb}{0,.145,.698}
    \definecolor{linkcolor}{rgb}{.71,0.21,0.01}
    \definecolor{citecolor}{rgb}{.12,.54,.11}

    % ANSI colors
    \definecolor{ansi-black}{HTML}{3E424D}
    \definecolor{ansi-black-intense}{HTML}{282C36}
    \definecolor{ansi-red}{HTML}{E75C58}
    \definecolor{ansi-red-intense}{HTML}{B22B31}
    \definecolor{ansi-green}{HTML}{00A250}
    \definecolor{ansi-green-intense}{HTML}{007427}
    \definecolor{ansi-yellow}{HTML}{DDB62B}
    \definecolor{ansi-yellow-intense}{HTML}{B27D12}
    \definecolor{ansi-blue}{HTML}{208FFB}
    \definecolor{ansi-blue-intense}{HTML}{0065CA}
    \definecolor{ansi-magenta}{HTML}{D160C4}
    \definecolor{ansi-magenta-intense}{HTML}{A03196}
    \definecolor{ansi-cyan}{HTML}{60C6C8}
    \definecolor{ansi-cyan-intense}{HTML}{258F8F}
    \definecolor{ansi-white}{HTML}{C5C1B4}
    \definecolor{ansi-white-intense}{HTML}{A1A6B2}

    % commands and environments needed by pandoc snippets
    % extracted from the output of `pandoc -s`
    \providecommand{\tightlist}{%
      \setlength{\itemsep}{0pt}\setlength{\parskip}{0pt}}
    \DefineVerbatimEnvironment{Highlighting}{Verbatim}{commandchars=\\\{\}}
    % Add ',fontsize=\small' for more characters per line
    \newenvironment{Shaded}{}{}
    \newcommand{\KeywordTok}[1]{\textcolor[rgb]{0.00,0.44,0.13}{\textbf{{#1}}}}
    \newcommand{\DataTypeTok}[1]{\textcolor[rgb]{0.56,0.13,0.00}{{#1}}}
    \newcommand{\DecValTok}[1]{\textcolor[rgb]{0.25,0.63,0.44}{{#1}}}
    \newcommand{\BaseNTok}[1]{\textcolor[rgb]{0.25,0.63,0.44}{{#1}}}
    \newcommand{\FloatTok}[1]{\textcolor[rgb]{0.25,0.63,0.44}{{#1}}}
    \newcommand{\CharTok}[1]{\textcolor[rgb]{0.25,0.44,0.63}{{#1}}}
    \newcommand{\StringTok}[1]{\textcolor[rgb]{0.25,0.44,0.63}{{#1}}}
    \newcommand{\CommentTok}[1]{\textcolor[rgb]{0.38,0.63,0.69}{\textit{{#1}}}}
    \newcommand{\OtherTok}[1]{\textcolor[rgb]{0.00,0.44,0.13}{{#1}}}
    \newcommand{\AlertTok}[1]{\textcolor[rgb]{1.00,0.00,0.00}{\textbf{{#1}}}}
    \newcommand{\FunctionTok}[1]{\textcolor[rgb]{0.02,0.16,0.49}{{#1}}}
    \newcommand{\RegionMarkerTok}[1]{{#1}}
    \newcommand{\ErrorTok}[1]{\textcolor[rgb]{1.00,0.00,0.00}{\textbf{{#1}}}}
    \newcommand{\NormalTok}[1]{{#1}}
    
    % Additional commands for more recent versions of Pandoc
    \newcommand{\ConstantTok}[1]{\textcolor[rgb]{0.53,0.00,0.00}{{#1}}}
    \newcommand{\SpecialCharTok}[1]{\textcolor[rgb]{0.25,0.44,0.63}{{#1}}}
    \newcommand{\VerbatimStringTok}[1]{\textcolor[rgb]{0.25,0.44,0.63}{{#1}}}
    \newcommand{\SpecialStringTok}[1]{\textcolor[rgb]{0.73,0.40,0.53}{{#1}}}
    \newcommand{\ImportTok}[1]{{#1}}
    \newcommand{\DocumentationTok}[1]{\textcolor[rgb]{0.73,0.13,0.13}{\textit{{#1}}}}
    \newcommand{\AnnotationTok}[1]{\textcolor[rgb]{0.38,0.63,0.69}{\textbf{\textit{{#1}}}}}
    \newcommand{\CommentVarTok}[1]{\textcolor[rgb]{0.38,0.63,0.69}{\textbf{\textit{{#1}}}}}
    \newcommand{\VariableTok}[1]{\textcolor[rgb]{0.10,0.09,0.49}{{#1}}}
    \newcommand{\ControlFlowTok}[1]{\textcolor[rgb]{0.00,0.44,0.13}{\textbf{{#1}}}}
    \newcommand{\OperatorTok}[1]{\textcolor[rgb]{0.40,0.40,0.40}{{#1}}}
    \newcommand{\BuiltInTok}[1]{{#1}}
    \newcommand{\ExtensionTok}[1]{{#1}}
    \newcommand{\PreprocessorTok}[1]{\textcolor[rgb]{0.74,0.48,0.00}{{#1}}}
    \newcommand{\AttributeTok}[1]{\textcolor[rgb]{0.49,0.56,0.16}{{#1}}}
    \newcommand{\InformationTok}[1]{\textcolor[rgb]{0.38,0.63,0.69}{\textbf{\textit{{#1}}}}}
    \newcommand{\WarningTok}[1]{\textcolor[rgb]{0.38,0.63,0.69}{\textbf{\textit{{#1}}}}}
    
    
    % Define a nice break command that doesn't care if a line doesn't already
    % exist.
    \def\br{\hspace*{\fill} \\* }
    % Math Jax compatability definitions
    \def\gt{>}
    \def\lt{<}
    % Document parameters
    \title{summary-02}
    
    
    

    % Pygments definitions
    
\makeatletter
\def\PY@reset{\let\PY@it=\relax \let\PY@bf=\relax%
    \let\PY@ul=\relax \let\PY@tc=\relax%
    \let\PY@bc=\relax \let\PY@ff=\relax}
\def\PY@tok#1{\csname PY@tok@#1\endcsname}
\def\PY@toks#1+{\ifx\relax#1\empty\else%
    \PY@tok{#1}\expandafter\PY@toks\fi}
\def\PY@do#1{\PY@bc{\PY@tc{\PY@ul{%
    \PY@it{\PY@bf{\PY@ff{#1}}}}}}}
\def\PY#1#2{\PY@reset\PY@toks#1+\relax+\PY@do{#2}}

\expandafter\def\csname PY@tok@w\endcsname{\def\PY@tc##1{\textcolor[rgb]{0.73,0.73,0.73}{##1}}}
\expandafter\def\csname PY@tok@c\endcsname{\let\PY@it=\textit\def\PY@tc##1{\textcolor[rgb]{0.25,0.50,0.50}{##1}}}
\expandafter\def\csname PY@tok@cp\endcsname{\def\PY@tc##1{\textcolor[rgb]{0.74,0.48,0.00}{##1}}}
\expandafter\def\csname PY@tok@k\endcsname{\let\PY@bf=\textbf\def\PY@tc##1{\textcolor[rgb]{0.00,0.50,0.00}{##1}}}
\expandafter\def\csname PY@tok@kp\endcsname{\def\PY@tc##1{\textcolor[rgb]{0.00,0.50,0.00}{##1}}}
\expandafter\def\csname PY@tok@kt\endcsname{\def\PY@tc##1{\textcolor[rgb]{0.69,0.00,0.25}{##1}}}
\expandafter\def\csname PY@tok@o\endcsname{\def\PY@tc##1{\textcolor[rgb]{0.40,0.40,0.40}{##1}}}
\expandafter\def\csname PY@tok@ow\endcsname{\let\PY@bf=\textbf\def\PY@tc##1{\textcolor[rgb]{0.67,0.13,1.00}{##1}}}
\expandafter\def\csname PY@tok@nb\endcsname{\def\PY@tc##1{\textcolor[rgb]{0.00,0.50,0.00}{##1}}}
\expandafter\def\csname PY@tok@nf\endcsname{\def\PY@tc##1{\textcolor[rgb]{0.00,0.00,1.00}{##1}}}
\expandafter\def\csname PY@tok@nc\endcsname{\let\PY@bf=\textbf\def\PY@tc##1{\textcolor[rgb]{0.00,0.00,1.00}{##1}}}
\expandafter\def\csname PY@tok@nn\endcsname{\let\PY@bf=\textbf\def\PY@tc##1{\textcolor[rgb]{0.00,0.00,1.00}{##1}}}
\expandafter\def\csname PY@tok@ne\endcsname{\let\PY@bf=\textbf\def\PY@tc##1{\textcolor[rgb]{0.82,0.25,0.23}{##1}}}
\expandafter\def\csname PY@tok@nv\endcsname{\def\PY@tc##1{\textcolor[rgb]{0.10,0.09,0.49}{##1}}}
\expandafter\def\csname PY@tok@no\endcsname{\def\PY@tc##1{\textcolor[rgb]{0.53,0.00,0.00}{##1}}}
\expandafter\def\csname PY@tok@nl\endcsname{\def\PY@tc##1{\textcolor[rgb]{0.63,0.63,0.00}{##1}}}
\expandafter\def\csname PY@tok@ni\endcsname{\let\PY@bf=\textbf\def\PY@tc##1{\textcolor[rgb]{0.60,0.60,0.60}{##1}}}
\expandafter\def\csname PY@tok@na\endcsname{\def\PY@tc##1{\textcolor[rgb]{0.49,0.56,0.16}{##1}}}
\expandafter\def\csname PY@tok@nt\endcsname{\let\PY@bf=\textbf\def\PY@tc##1{\textcolor[rgb]{0.00,0.50,0.00}{##1}}}
\expandafter\def\csname PY@tok@nd\endcsname{\def\PY@tc##1{\textcolor[rgb]{0.67,0.13,1.00}{##1}}}
\expandafter\def\csname PY@tok@s\endcsname{\def\PY@tc##1{\textcolor[rgb]{0.73,0.13,0.13}{##1}}}
\expandafter\def\csname PY@tok@sd\endcsname{\let\PY@it=\textit\def\PY@tc##1{\textcolor[rgb]{0.73,0.13,0.13}{##1}}}
\expandafter\def\csname PY@tok@si\endcsname{\let\PY@bf=\textbf\def\PY@tc##1{\textcolor[rgb]{0.73,0.40,0.53}{##1}}}
\expandafter\def\csname PY@tok@se\endcsname{\let\PY@bf=\textbf\def\PY@tc##1{\textcolor[rgb]{0.73,0.40,0.13}{##1}}}
\expandafter\def\csname PY@tok@sr\endcsname{\def\PY@tc##1{\textcolor[rgb]{0.73,0.40,0.53}{##1}}}
\expandafter\def\csname PY@tok@ss\endcsname{\def\PY@tc##1{\textcolor[rgb]{0.10,0.09,0.49}{##1}}}
\expandafter\def\csname PY@tok@sx\endcsname{\def\PY@tc##1{\textcolor[rgb]{0.00,0.50,0.00}{##1}}}
\expandafter\def\csname PY@tok@m\endcsname{\def\PY@tc##1{\textcolor[rgb]{0.40,0.40,0.40}{##1}}}
\expandafter\def\csname PY@tok@gh\endcsname{\let\PY@bf=\textbf\def\PY@tc##1{\textcolor[rgb]{0.00,0.00,0.50}{##1}}}
\expandafter\def\csname PY@tok@gu\endcsname{\let\PY@bf=\textbf\def\PY@tc##1{\textcolor[rgb]{0.50,0.00,0.50}{##1}}}
\expandafter\def\csname PY@tok@gd\endcsname{\def\PY@tc##1{\textcolor[rgb]{0.63,0.00,0.00}{##1}}}
\expandafter\def\csname PY@tok@gi\endcsname{\def\PY@tc##1{\textcolor[rgb]{0.00,0.63,0.00}{##1}}}
\expandafter\def\csname PY@tok@gr\endcsname{\def\PY@tc##1{\textcolor[rgb]{1.00,0.00,0.00}{##1}}}
\expandafter\def\csname PY@tok@ge\endcsname{\let\PY@it=\textit}
\expandafter\def\csname PY@tok@gs\endcsname{\let\PY@bf=\textbf}
\expandafter\def\csname PY@tok@gp\endcsname{\let\PY@bf=\textbf\def\PY@tc##1{\textcolor[rgb]{0.00,0.00,0.50}{##1}}}
\expandafter\def\csname PY@tok@go\endcsname{\def\PY@tc##1{\textcolor[rgb]{0.53,0.53,0.53}{##1}}}
\expandafter\def\csname PY@tok@gt\endcsname{\def\PY@tc##1{\textcolor[rgb]{0.00,0.27,0.87}{##1}}}
\expandafter\def\csname PY@tok@err\endcsname{\def\PY@bc##1{\setlength{\fboxsep}{0pt}\fcolorbox[rgb]{1.00,0.00,0.00}{1,1,1}{\strut ##1}}}
\expandafter\def\csname PY@tok@kc\endcsname{\let\PY@bf=\textbf\def\PY@tc##1{\textcolor[rgb]{0.00,0.50,0.00}{##1}}}
\expandafter\def\csname PY@tok@kd\endcsname{\let\PY@bf=\textbf\def\PY@tc##1{\textcolor[rgb]{0.00,0.50,0.00}{##1}}}
\expandafter\def\csname PY@tok@kn\endcsname{\let\PY@bf=\textbf\def\PY@tc##1{\textcolor[rgb]{0.00,0.50,0.00}{##1}}}
\expandafter\def\csname PY@tok@kr\endcsname{\let\PY@bf=\textbf\def\PY@tc##1{\textcolor[rgb]{0.00,0.50,0.00}{##1}}}
\expandafter\def\csname PY@tok@bp\endcsname{\def\PY@tc##1{\textcolor[rgb]{0.00,0.50,0.00}{##1}}}
\expandafter\def\csname PY@tok@fm\endcsname{\def\PY@tc##1{\textcolor[rgb]{0.00,0.00,1.00}{##1}}}
\expandafter\def\csname PY@tok@vc\endcsname{\def\PY@tc##1{\textcolor[rgb]{0.10,0.09,0.49}{##1}}}
\expandafter\def\csname PY@tok@vg\endcsname{\def\PY@tc##1{\textcolor[rgb]{0.10,0.09,0.49}{##1}}}
\expandafter\def\csname PY@tok@vi\endcsname{\def\PY@tc##1{\textcolor[rgb]{0.10,0.09,0.49}{##1}}}
\expandafter\def\csname PY@tok@vm\endcsname{\def\PY@tc##1{\textcolor[rgb]{0.10,0.09,0.49}{##1}}}
\expandafter\def\csname PY@tok@sa\endcsname{\def\PY@tc##1{\textcolor[rgb]{0.73,0.13,0.13}{##1}}}
\expandafter\def\csname PY@tok@sb\endcsname{\def\PY@tc##1{\textcolor[rgb]{0.73,0.13,0.13}{##1}}}
\expandafter\def\csname PY@tok@sc\endcsname{\def\PY@tc##1{\textcolor[rgb]{0.73,0.13,0.13}{##1}}}
\expandafter\def\csname PY@tok@dl\endcsname{\def\PY@tc##1{\textcolor[rgb]{0.73,0.13,0.13}{##1}}}
\expandafter\def\csname PY@tok@s2\endcsname{\def\PY@tc##1{\textcolor[rgb]{0.73,0.13,0.13}{##1}}}
\expandafter\def\csname PY@tok@sh\endcsname{\def\PY@tc##1{\textcolor[rgb]{0.73,0.13,0.13}{##1}}}
\expandafter\def\csname PY@tok@s1\endcsname{\def\PY@tc##1{\textcolor[rgb]{0.73,0.13,0.13}{##1}}}
\expandafter\def\csname PY@tok@mb\endcsname{\def\PY@tc##1{\textcolor[rgb]{0.40,0.40,0.40}{##1}}}
\expandafter\def\csname PY@tok@mf\endcsname{\def\PY@tc##1{\textcolor[rgb]{0.40,0.40,0.40}{##1}}}
\expandafter\def\csname PY@tok@mh\endcsname{\def\PY@tc##1{\textcolor[rgb]{0.40,0.40,0.40}{##1}}}
\expandafter\def\csname PY@tok@mi\endcsname{\def\PY@tc##1{\textcolor[rgb]{0.40,0.40,0.40}{##1}}}
\expandafter\def\csname PY@tok@il\endcsname{\def\PY@tc##1{\textcolor[rgb]{0.40,0.40,0.40}{##1}}}
\expandafter\def\csname PY@tok@mo\endcsname{\def\PY@tc##1{\textcolor[rgb]{0.40,0.40,0.40}{##1}}}
\expandafter\def\csname PY@tok@ch\endcsname{\let\PY@it=\textit\def\PY@tc##1{\textcolor[rgb]{0.25,0.50,0.50}{##1}}}
\expandafter\def\csname PY@tok@cm\endcsname{\let\PY@it=\textit\def\PY@tc##1{\textcolor[rgb]{0.25,0.50,0.50}{##1}}}
\expandafter\def\csname PY@tok@cpf\endcsname{\let\PY@it=\textit\def\PY@tc##1{\textcolor[rgb]{0.25,0.50,0.50}{##1}}}
\expandafter\def\csname PY@tok@c1\endcsname{\let\PY@it=\textit\def\PY@tc##1{\textcolor[rgb]{0.25,0.50,0.50}{##1}}}
\expandafter\def\csname PY@tok@cs\endcsname{\let\PY@it=\textit\def\PY@tc##1{\textcolor[rgb]{0.25,0.50,0.50}{##1}}}

\def\PYZbs{\char`\\}
\def\PYZus{\char`\_}
\def\PYZob{\char`\{}
\def\PYZcb{\char`\}}
\def\PYZca{\char`\^}
\def\PYZam{\char`\&}
\def\PYZlt{\char`\<}
\def\PYZgt{\char`\>}
\def\PYZsh{\char`\#}
\def\PYZpc{\char`\%}
\def\PYZdl{\char`\$}
\def\PYZhy{\char`\-}
\def\PYZsq{\char`\'}
\def\PYZdq{\char`\"}
\def\PYZti{\char`\~}
% for compatibility with earlier versions
\def\PYZat{@}
\def\PYZlb{[}
\def\PYZrb{]}
\makeatother


    % Exact colors from NB
    \definecolor{incolor}{rgb}{0.0, 0.0, 0.5}
    \definecolor{outcolor}{rgb}{0.545, 0.0, 0.0}



    
    % Prevent overflowing lines due to hard-to-break entities
    \sloppy 
    % Setup hyperref package
    \hypersetup{
      breaklinks=true,  % so long urls are correctly broken across lines
      colorlinks=true,
      urlcolor=urlcolor,
      linkcolor=linkcolor,
      citecolor=citecolor,
      }
    % Slightly bigger margins than the latex defaults
    
    \geometry{verbose,tmargin=1in,bmargin=1in,lmargin=1in,rmargin=1in}
    
    

    \begin{document}
    
    
    \maketitle
    
    

    
    \begin{Verbatim}[commandchars=\\\{\}]
{\color{incolor}In [{\color{incolor}138}]:} \PY{k+kp}{rm}\PY{p}{(}\PY{k+kt}{list} \PY{o}{=} \PY{k+kp}{ls}\PY{p}{(}\PY{p}{)}\PY{p}{)}
          \PY{k+kn}{library}\PY{p}{(}MASS\PY{p}{)}
          \PY{k+kn}{library}\PY{p}{(}ggplot2\PY{p}{)}
          \PY{k+kn}{library}\PY{p}{(}gridExtra\PY{p}{)}
          \PY{k+kn}{library}\PY{p}{(}carData\PY{p}{)}
          \PY{k+kn}{library}\PY{p}{(}car\PY{p}{)}
          \PY{k+kn}{library}\PY{p}{(}TSA\PY{p}{)}
          \PY{k+kn}{library}\PY{p}{(}jtools\PY{p}{)}
          \PY{k+kn}{library}\PY{p}{(}png\PY{p}{)}
          \PY{k+kn}{library}\PY{p}{(}grid\PY{p}{)}
          dat \PY{o}{=} read.csv\PY{p}{(}\PY{l+s}{\PYZdq{}}\PY{l+s}{BodyFat.csv\PYZdq{}}\PY{p}{,} row.names \PY{o}{=} \PY{l+m}{1}\PY{p}{)}
          dat\PY{o}{\PYZdl{}}WEIGHT\PY{o}{=}\PY{k+kp}{round}\PY{p}{(}dat\PY{o}{\PYZdl{}}WEIGHT\PY{o}{*}\PY{l+m}{0.45359237}\PY{p}{,}\PY{l+m}{2}\PY{p}{)}
          dat\PY{o}{\PYZdl{}}HEIGHT\PY{o}{=}\PY{k+kp}{round}\PY{p}{(}dat\PY{o}{\PYZdl{}}HEIGHT\PY{o}{*}\PY{l+m}{2.54}\PY{p}{,}\PY{l+m}{2}\PY{p}{)}
          bodyfat48\PY{o}{=}\PY{l+m}{495}\PY{o}{/}\PY{l+m}{1.0665}\PY{l+m}{\PYZhy{}450} \PY{c+c1}{\PYZsh{}bodyfay wrong}
          bodyfat96\PY{o}{=}\PY{l+m}{495}\PY{o}{/}\PY{l+m}{1.0991}\PY{l+m}{\PYZhy{}450} \PY{c+c1}{\PYZsh{}density wrong}
          bodyfat76\PY{o}{=}\PY{l+m}{495}\PY{o}{/}\PY{l+m}{1.0666}\PY{l+m}{\PYZhy{}450} \PY{c+c1}{\PYZsh{}density wrong}
\end{Verbatim}


    \subparagraph{1}\label{section}

\subparagraph{1}\label{section-1}

\subparagraph{1}\label{section-2}

\subparagraph{1}\label{section-3}

\subparagraph{1}\label{section-4}

\subparagraph{1}\label{section-5}

\subparagraph{1}\label{section-6}

\subparagraph{1}\label{section-7}

\subparagraph{1}\label{section-8}

    \subsection{Introduction}\label{introduction}

    In this project, we build a two-term nonlinear model to estimate
percentage of males' body fat based on a real data set of 252 men. Our
model has a good balance between simplicity, robustness, accuracy and
precise. We also build a body fat calculator written by Rshiny.

    \subsection{Data Cleaning}\label{data-cleaning}

    We first take a glimpse of the raw data to see if they agree with the
realistic background, whether there is any weird record, outlier or
typo. Since the unit of most of the variables is centimeter, to unify
units, we convert the unit of height to centimeter. And to make it
easier to calculate bmi (ADIPOSITY), also change the unit of weight to
kilogram. Than the formula of bmi
becomes:\(BMI = \frac{WEIGHT}{{HEIGHT/100}^{2}}\)

    \begin{longtable}[]{@{}lllll@{}}
\toprule
\textbar{}MIN & 1st Qu & Median & Mean & 3rd Qu\tabularnewline
\midrule
\endhead
BODYFAT & 0.00 & 12.80 & 19.00 & 18.94\tabularnewline
WEIGHT & 53.75 & 72.12 & 80.06 & 81.16\tabularnewline
HEIGHT & 74.93 & 173.35 & 177.80 & 178.18\tabularnewline
AGE & 22.00 & 35.75 & 43.00 & 44.88\tabularnewline
\bottomrule
\end{longtable}

    \begin{Verbatim}[commandchars=\\\{\}]
{\color{incolor}In [{\color{incolor}139}]:} \PY{k+kp}{options}\PY{p}{(}repr.plot.width\PY{o}{=}\PY{l+m}{1.8}\PY{p}{,}repr.plot.height\PY{o}{=}\PY{l+m}{1.2}\PY{p}{,}repr.plot.pointsize\PY{o}{=}\PY{l+m}{2.5}\PY{p}{)}\PY{p}{;}par\PY{p}{(}mfrow\PY{o}{=}\PY{k+kt}{c}\PY{p}{(}\PY{l+m}{1}\PY{p}{,}\PY{l+m}{2}\PY{p}{)}\PY{p}{)}\PY{p}{;}boxplot\PY{p}{(}dat\PY{o}{\PYZdl{}}WEIGHT\PY{p}{,}main\PY{o}{=}\PY{l+s}{\PYZdq{}}\PY{l+s}{WEIGHT\PYZdq{}}\PY{p}{)}\PY{p}{;}boxplot\PY{p}{(}dat\PY{o}{\PYZdl{}}HEIGHT\PY{p}{,}main\PY{o}{=}\PY{l+s}{\PYZdq{}}\PY{l+s}{HEIGHT\PYZdq{}}\PY{p}{)}
\end{Verbatim}


    \begin{center}
    \adjustimage{max size={0.9\linewidth}{0.9\paperheight}}{output_7_0.png}
    \end{center}
    { \hspace*{\fill} \\}
    
    By looking at the dataset summary and boxplots, we detect four weird
records: \#39 weights extremely heavy, \#42 is too short, \#79 much
older than other men, \#182 has zero bodyfat. Then we tried to use the
variable values that seem to be true to predict the corresponding
outlier variable for these three persons. However, we are still not sure
about whether the four outliers were wrong values or just extreme
values, so we check consistency of BMI versus HEIGHT and WEIGHT, also
the consistency between density and bodyfat.

    \paragraph{Consistency of BMI versus HEIGHT and
WEIGHT}\label{consistency-of-bmi-versus-height-and-weight}

    \begin{longtable}[]{@{}lllll@{}}
\toprule
point & extrem value & solution &\tabularnewline
\midrule
\endhead
39 & weight 164.72 kg & not change &\tabularnewline
42 & height 74.93 cm & change to 176.35 &\tabularnewline
79 & age 81 years old & not change &\tabularnewline
\bottomrule
\end{longtable}

    We change the height of \#42 but retain \#79 and the weight of \#39
because it seems that the 79th is a normal thin old man, the 39th is a
very heavy man which follows the bmi equation. We'd better keep more
information.

    \begin{Verbatim}[commandchars=\\\{\}]
{\color{incolor}In [{\color{incolor}140}]:} dat\PY{o}{\PYZdl{}}HEIGHT\PY{p}{[}\PY{l+m}{42}\PY{p}{]}\PY{o}{=}\PY{k+kp}{round}\PY{p}{(}\PY{k+kp}{sqrt}\PY{p}{(}\PY{l+m}{92.99}\PY{o}{/}\PY{l+m}{29.9}\PY{p}{)}\PY{o}{*}\PY{l+m}{100}\PY{p}{,}\PY{l+m}{2}\PY{p}{)}
\end{Verbatim}


    \paragraph{Consistency between DENSITY and
BODYFAT}\label{consistency-between-density-and-bodyfat}

    \begin{Verbatim}[commandchars=\\\{\}]
{\color{incolor}In [{\color{incolor}141}]:} \PY{k+kp}{options}\PY{p}{(}repr.plot.width\PY{o}{=}\PY{l+m}{2}\PY{p}{,}repr.plot.height\PY{o}{=}\PY{l+m}{2}\PY{p}{,}repr.plot.pointsize\PY{o}{=}\PY{l+m}{7}\PY{p}{)}\PY{p}{;}par\PY{p}{(}mfrow\PY{o}{=}\PY{k+kt}{c}\PY{p}{(}\PY{l+m}{1}\PY{p}{,}\PY{l+m}{1}\PY{p}{)}\PY{p}{)}\PY{p}{;}plot\PY{p}{(}dat\PY{o}{\PYZdl{}}BODYFAT\PY{o}{\PYZti{}}\PY{k+kp}{I}\PY{p}{(}dat\PY{o}{\PYZdl{}}DENSITY\PY{o}{\PYZca{}}\PY{p}{(}\PY{l+m}{\PYZhy{}1}\PY{p}{)}\PY{p}{)}\PY{p}{,}xlim\PY{o}{=}\PY{k+kt}{c}\PY{p}{(}\PY{l+m}{0.90}\PY{p}{,}\PY{l+m}{1.01}\PY{p}{)}\PY{p}{,}ylab\PY{o}{=}\PY{l+s}{\PYZdq{}}\PY{l+s}{Bodyfat\PYZdq{}}\PY{p}{,}xlab\PY{o}{=}\PY{l+s}{\PYZdq{}}\PY{l+s}{1/Density\PYZdq{}}\PY{p}{,}col\PY{o}{=}\PY{l+s}{\PYZdq{}}\PY{l+s}{skyblue\PYZdq{}}\PY{p}{,}pch\PY{o}{=}\PY{l+m}{19}\PY{p}{,}cex\PY{o}{=}\PY{l+m}{0.7}\PY{p}{,}main\PY{o}{=}\PY{l+s}{\PYZdq{}}\PY{l+s}{Bodyfat vs 1/Density\PYZdq{}}\PY{p}{)}\PY{p}{;}x\PY{o}{=}\PY{k+kp}{seq}\PY{p}{(}\PY{l+m}{0.9}\PY{p}{,}\PY{l+m}{1.01}\PY{p}{,}\PY{l+m}{0.01}\PY{p}{)}\PY{p}{;}lines\PY{p}{(}x\PY{p}{,}y\PY{o}{=}x\PY{o}{*}\PY{l+m}{495}\PY{l+m}{\PYZhy{}450}\PY{p}{,}col\PY{o}{=}\PY{l+s}{\PYZdq{}}\PY{l+s}{navyblue\PYZdq{}}\PY{p}{,}lwd\PY{o}{=}\PY{l+m}{1.2}\PY{p}{)}\PY{p}{;}text\PY{p}{(}\PY{l+m}{0.99}\PY{p}{,}\PY{l+m}{43}\PY{p}{,}\PY{l+s}{\PYZdq{}}\PY{l+s}{y=495/x\PYZhy{}450\PYZdq{}}\PY{p}{,}col\PY{o}{=}\PY{l+s}{\PYZdq{}}\PY{l+s}{navyblue\PYZdq{}}\PY{p}{,}cex\PY{o}{=}\PY{l+m}{1}\PY{p}{)}\PY{p}{;}text\PY{p}{(}\PY{l+m}{1}\PY{o}{/}\PY{l+m}{1.0991}\PY{p}{,}\PY{l+m}{18.3}\PY{p}{,}\PY{l+s}{\PYZdq{}}\PY{l+s}{96\PYZdq{}}\PY{p}{,}col\PY{o}{=}\PY{l+s}{\PYZdq{}}\PY{l+s}{navyblue\PYZdq{}}\PY{p}{,}cex\PY{o}{=}\PY{l+m}{1}\PY{p}{)}\PY{p}{;}text\PY{p}{(}\PY{l+m}{1}\PY{o}{/}\PY{l+m}{1.0665}\PY{p}{,}\PY{l+m}{5.4}\PY{p}{,}\PY{l+s}{\PYZdq{}}\PY{l+s}{48\PYZdq{}}\PY{p}{,}col\PY{o}{=}\PY{l+s}{\PYZdq{}}\PY{l+s}{navyblue\PYZdq{}}\PY{p}{,}cex\PY{o}{=}\PY{l+m}{1}\PY{p}{)}\PY{p}{;}text\PY{p}{(}\PY{l+m}{1}\PY{o}{/}\PY{l+m}{1.0666}\PY{p}{,}\PY{l+m}{19.3}\PY{p}{,}\PY{l+s}{\PYZdq{}}\PY{l+s}{76\PYZdq{}}\PY{p}{,}col\PY{o}{=}\PY{l+s}{\PYZdq{}}\PY{l+s}{navyblue\PYZdq{}}\PY{p}{,}cex\PY{o}{=}\PY{l+m}{1}\PY{p}{)}
\end{Verbatim}


    \begin{center}
    \adjustimage{max size={0.9\linewidth}{0.9\paperheight}}{output_14_0.png}
    \end{center}
    { \hspace*{\fill} \\}
    
    We delete \#182 variable because it can't be imputed by density, the
value of 495/1.1089 - 450 is less than zero. Also, noticing there are
three data points whose bodyfat conflict with density use the siri's
formula. Since we are not sure of the bodyfat and density which one is
wrong, we search for a bodyfat calculator as reference to re-calculate
these three bodyfat usig Weight and Waist Circumference(Abdomen) two
variables.

    \begin{longtable}[]{@{}lllll@{}}
\toprule
point & bodyfat/density wrong & solution &\tabularnewline
\midrule
\endhead
48 & bodyfat wrong & bodyfat change to 14.14 &\tabularnewline
96 & density wrong & keep bodyfat &\tabularnewline
76 & density wrong & keep bodyfat &\tabularnewline
182 & bodyfat wrong & delete &\tabularnewline
\bottomrule
\end{longtable}

    After checking the bodyfat calculated by siri's formula and by
calculator, we got the conclusion that only point 48 has wrong bodyfat
and we use siri's formula to impute it.

    \begin{Verbatim}[commandchars=\\\{\}]
{\color{incolor}In [{\color{incolor}142}]:} dat\PY{o}{=}dat\PY{p}{[}\PY{l+m}{\PYZhy{}182}\PY{p}{,}\PY{p}{]}\PY{p}{;}dat\PY{o}{\PYZdl{}}BODYFAT\PY{p}{[}\PY{l+m}{48}\PY{p}{]}\PY{o}{=}bodyfat48\PY{p}{;}dat\PY{o}{=}dat\PY{p}{[}\PY{p}{,}\PY{l+m}{\PYZhy{}2}\PY{p}{]} \PY{c+c1}{\PYZsh{}delete density}
\end{Verbatim}


    \subsection{Variable Selection}\label{variable-selection}

    According to the forward variables selection based on AIC above,
considering the simplicity, the best models with different number of
variables, 1 to 4, are as below:

    \begin{Verbatim}[commandchars=\\\{\}]
{\color{incolor}In [{\color{incolor}143}]:} m1\PY{o}{\PYZlt{}\PYZhy{}}lm\PY{p}{(}BODYFAT\PY{o}{\PYZti{}}ABDOMEN\PY{p}{,}data\PY{o}{=}dat\PY{p}{)}\PY{p}{;}m2\PY{o}{\PYZlt{}\PYZhy{}}lm\PY{p}{(}BODYFAT\PY{o}{\PYZti{}}ABDOMEN\PY{o}{+}WEIGHT\PY{p}{,}data\PY{o}{=}dat\PY{p}{)}\PY{p}{;}m3\PY{o}{\PYZlt{}\PYZhy{}}lm\PY{p}{(}BODYFAT\PY{o}{\PYZti{}}ABDOMEN\PY{o}{+}WEIGHT\PY{o}{+}WRIST\PY{p}{,}data\PY{o}{=}dat\PY{p}{)}\PY{p}{;}m4\PY{o}{\PYZlt{}\PYZhy{}}lm\PY{p}{(}BODYFAT\PY{o}{\PYZti{}}ABDOMEN\PY{o}{+}WEIGHT\PY{o}{+}WRIST\PY{o}{+}FOREARM\PY{p}{,}data\PY{o}{=}dat\PY{p}{)}
\end{Verbatim}


    \begin{longtable}[]{@{}ll@{}}
\toprule
model name & model\tabularnewline
\midrule
\endhead
m1 & BODYFAT\textasciitilde{}ABDOMEN\tabularnewline
m2 & BODYFAT\textasciitilde{}ABDOMEN+WEIGHT\tabularnewline
m3 & BODYFAT\textasciitilde{}ABDOMEN+WEIGHT+WRIST\tabularnewline
m4 & BODYFAT\textasciitilde{}ABDOMEN+WEIGHT+WRIST+FOREARM\tabularnewline
\bottomrule
\end{longtable}

    The variables selected are mainly Abdomen, Weight and Wrist and we want
to look into their relationships and give a intuitive analysis based on
data visualization.

    \begin{Verbatim}[commandchars=\\\{\}]
{\color{incolor}In [{\color{incolor}144}]:} \PY{k+kp}{options}\PY{p}{(}repr.plot.width\PY{o}{=}\PY{l+m}{5}\PY{p}{,}repr.plot.height\PY{o}{=}\PY{l+m}{2.3}\PY{p}{,}repr.plot.pointsize\PY{o}{=}\PY{l+m}{6}\PY{p}{)}\PY{p}{;}par\PY{p}{(}mfrow\PY{o}{=}\PY{k+kt}{c}\PY{p}{(}\PY{l+m}{2}\PY{p}{,}\PY{l+m}{2}\PY{p}{)}\PY{p}{)}\PY{p}{;}plot\PY{p}{(}dat\PY{o}{\PYZdl{}}BODYFAT\PY{o}{\PYZti{}}dat\PY{o}{\PYZdl{}}ABDOMEN\PY{p}{)}\PY{p}{;}abline\PY{p}{(}v\PY{o}{=}\PY{l+m}{116}\PY{p}{,}col\PY{o}{=}\PY{l+s}{\PYZdq{}}\PY{l+s}{skyblue\PYZdq{}}\PY{p}{,}lty\PY{o}{=}\PY{l+s}{\PYZdq{}}\PY{l+s}{dashed\PYZdq{}}\PY{p}{)}\PY{p}{;}text\PY{p}{(}\PY{l+m}{122}\PY{p}{,}\PY{l+m}{43.5}\PY{p}{,}\PY{l+s}{\PYZdq{}}\PY{l+s}{216\PYZdq{}}\PY{p}{,}col\PY{o}{=}\PY{l+s}{\PYZdq{}}\PY{l+s}{navyblue\PYZdq{}}\PY{p}{,}cex\PY{o}{=}\PY{l+m}{1}\PY{p}{)}\PY{p}{;}text\PY{p}{(}\PY{l+m}{146.5}\PY{p}{,}\PY{l+m}{32.5}\PY{p}{,}\PY{l+s}{\PYZdq{}}\PY{l+s}{39\PYZdq{}}\PY{p}{,}col\PY{o}{=}\PY{l+s}{\PYZdq{}}\PY{l+s}{navyblue\PYZdq{}}\PY{p}{,}cex\PY{o}{=}\PY{l+m}{1}\PY{p}{)}\PY{p}{;}plot\PY{p}{(}dat\PY{o}{\PYZdl{}}ABDOMEN\PY{o}{\PYZti{}}dat\PY{o}{\PYZdl{}}WEIGHT\PY{p}{)}\PY{p}{;}plot\PY{p}{(}dat\PY{o}{\PYZdl{}}BODYFAT\PY{o}{\PYZti{}}dat\PY{o}{\PYZdl{}}WRIST\PY{p}{)}\PY{p}{;}text\PY{p}{(}\PY{l+m}{18.4}\PY{p}{,}\PY{l+m}{43.5}\PY{p}{,}\PY{l+s}{\PYZdq{}}\PY{l+s}{216\PYZdq{}}\PY{p}{,}col\PY{o}{=}\PY{l+s}{\PYZdq{}}\PY{l+s}{navyblue\PYZdq{}}\PY{p}{,}cex\PY{o}{=}\PY{l+m}{1}\PY{p}{)}\PY{p}{;}text\PY{p}{(}\PY{l+m}{21.4}\PY{p}{,}\PY{l+m}{32.5}\PY{p}{,}\PY{l+s}{\PYZdq{}}\PY{l+s}{39\PYZdq{}}\PY{p}{,}col\PY{o}{=}\PY{l+s}{\PYZdq{}}\PY{l+s}{navyblue\PYZdq{}}\PY{p}{,}cex\PY{o}{=}\PY{l+m}{1}\PY{p}{)}\PY{p}{;}plot\PY{p}{(}dat\PY{o}{\PYZdl{}}ABDOMEN\PY{o}{\PYZti{}}dat\PY{o}{\PYZdl{}}WRIST\PY{p}{)}\PY{p}{;}abline\PY{p}{(}v\PY{o}{=}\PY{l+m}{20}\PY{p}{,}col\PY{o}{=}\PY{l+s}{\PYZdq{}}\PY{l+s}{skyblue\PYZdq{}}\PY{p}{,}lty\PY{o}{=}\PY{l+s}{\PYZdq{}}\PY{l+s}{dashed\PYZdq{}}\PY{p}{)}\PY{p}{;}text\PY{p}{(}\PY{l+m}{18.4}\PY{p}{,}\PY{l+m}{124}\PY{p}{,}\PY{l+s}{\PYZdq{}}\PY{l+s}{216\PYZdq{}}\PY{p}{,}col\PY{o}{=}\PY{l+s}{\PYZdq{}}\PY{l+s}{navyblue\PYZdq{}}\PY{p}{,}cex\PY{o}{=}\PY{l+m}{1}\PY{p}{)}\PY{p}{;}text\PY{p}{(}\PY{l+m}{21.4}\PY{p}{,}\PY{l+m}{146}\PY{p}{,}\PY{l+s}{\PYZdq{}}\PY{l+s}{39\PYZdq{}}\PY{p}{,}col\PY{o}{=}\PY{l+s}{\PYZdq{}}\PY{l+s}{navyblue\PYZdq{}}\PY{p}{,}cex\PY{o}{=}\PY{l+m}{1}\PY{p}{)}
\end{Verbatim}


    \begin{center}
    \adjustimage{max size={0.9\linewidth}{0.9\paperheight}}{output_24_0.png}
    \end{center}
    { \hspace*{\fill} \\}
    
    There exists an obvious linear relationship between Abdomen and Weight,
so the second model (m2) is unreliable because of the multicollinearity.
The vif test and high correlation (0.89) between the two variables also
support this statement. The fourth model (m4) has the same problem of
multicollinearity.

    \begin{longtable}[]{@{}lllll@{}}
\toprule
model & ABDOMEN & WEIGHT & WRIST & FOREARM\tabularnewline
\midrule
\endhead
m2 vif & 4.73 & 4.73 & &\tabularnewline
m4 vif & 4.86 & 7.04 & 2.27 & 1.79\tabularnewline
\bottomrule
\end{longtable}

    Because our goal is to find out a simple and accurate model, we first
put Model 4 aside and try to make some improvements on Model 2 and 3.

    \subsection{Model Modification and
Improvement}\label{model-modification-and-improvement}

    \paragraph{Ridge regression}\label{ridge-regression}

In order to eliminate multicollinearity of m2 model,we use ridge
regression and calculate MSE.

    \paragraph{Nonlinear model with interation
term}\label{nonlinear-model-with-interation-term}

    Althouth there's strong linearity between Abdomen and Bodyfat, Bodyfat
is influenced by other factors with higher abdomen level and it seems to
be a nonlinear relationship on the right side of the vertical line.
Wrist circumsference has no big influence on lower weight males, but
extremely large wrist circumsference will lead to large abdomen
circumsference and heavy weight. We can see the Abdomen increases
rapidly when Wrist is larger than 20. To further explore their mutual
influences, we turn to use interaction plots.

    \begin{Verbatim}[commandchars=\\\{\}]
{\color{incolor}In [{\color{incolor}145}]:} comb2pngs \PY{o}{\PYZlt{}\PYZhy{}} \PY{k+kr}{function}\PY{p}{(}imgs\PY{p}{,} bottom\PYZus{}text \PY{o}{=} \PY{k+kc}{NULL}\PY{p}{)}\PY{p}{\PYZob{}}
              img1 \PY{o}{\PYZlt{}\PYZhy{}}  grid\PY{o}{::}rasterGrob\PY{p}{(}as.raster\PY{p}{(}readPNG\PY{p}{(}imgs\PY{p}{[}\PY{l+m}{1}\PY{p}{]}\PY{p}{)}\PY{p}{)}\PY{p}{,}interpolate \PY{o}{=} \PY{k+kc}{FALSE}\PY{p}{)}
              img2 \PY{o}{\PYZlt{}\PYZhy{}}  grid\PY{o}{::}rasterGrob\PY{p}{(}as.raster\PY{p}{(}readPNG\PY{p}{(}imgs\PY{p}{[}\PY{l+m}{2}\PY{p}{]}\PY{p}{)}\PY{p}{)}\PY{p}{,}interpolate \PY{o}{=} \PY{k+kc}{FALSE}\PY{p}{)}
            grid.arrange\PY{p}{(}img1\PY{p}{,} img2\PY{p}{,} ncol \PY{o}{=} \PY{l+m}{2}\PY{p}{,} bottom \PY{o}{=} bottom\PYZus{}text\PY{p}{)}\PY{p}{\PYZcb{}}
          comb2pngs\PY{p}{(}\PY{k+kt}{c}\PY{p}{(}\PY{l+s}{\PYZdq{}}\PY{l+s}{picture6.png\PYZdq{}}\PY{p}{,} \PY{l+s}{\PYZdq{}}\PY{l+s}{picture7.png\PYZdq{}}\PY{p}{)}\PY{p}{)}
\end{Verbatim}


    \begin{center}
    \adjustimage{max size={0.9\linewidth}{0.9\paperheight}}{output_32_0.png}
    \end{center}
    { \hspace*{\fill} \\}
    
    Out of our expectation, at same Abdomen level, heavier males will have
lower bodyfat. Meanwhile, with same weight level, males with larger
wrist circumsference will have lower bodyfat. Therefore, we decide to
include the WEIGHT and WRIST interaction term into our model to
interprete their complicated relationships. And it turns out to be a
really good model.

    \subsection{Model Comparison}\label{model-comparison}

    \begin{longtable}[]{@{}lll@{}}
\toprule
Model & Adj.R-squared & MSE\tabularnewline
\midrule
\endhead
m2\_ridge & 0.7127 & 16.69\tabularnewline
m3 & 0.7185 & 16.16\tabularnewline
m5 & 0.7258 & 15.80\tabularnewline
\bottomrule
\end{longtable}

    m2\_ridge is improved from m2 and m5 is improved from m3. Table below
examines the accuracy of these three models by adjusted \(R^{2}\) and
MSE. As is shown, model 5 (m5) is much better according to both
criteria. Therefore, we choose model 5 as our final model.

\[ Bodyfat(\%) = -45.3249 + 0.9092\times Abdomen(cm) -0.0133\times Weight(kg)\times Wrist(cm)\]

Below is the coefficients of the model 5 and all of the coefficients are
significant.

    \begin{longtable}[]{@{}lllll@{}}
\toprule
/ & Estimate & Std.Error & t value & P value\tabularnewline
\midrule
\endhead
Intercept & -45.3249 & 2.573 & -17.62 & \textless{}2e-16\tabularnewline
ABDOMEN & 0.9092 & 0.047 & 19.42 & \textless{}2e-16\tabularnewline
WEIGHT:WRIST & -0.0133 & 0.002 & -8.22 & 1.14e-14\tabularnewline
\bottomrule
\end{longtable}

    \subsection{Model Diagnostic}\label{model-diagnostic}

    \begin{Verbatim}[commandchars=\\\{\}]
{\color{incolor}In [{\color{incolor}146}]:} m5\PY{o}{\PYZlt{}\PYZhy{}}lm\PY{p}{(}BODYFAT\PY{o}{\PYZti{}}ABDOMEN\PY{o}{+}WEIGHT\PY{o}{:}WRIST\PY{p}{,}data\PY{o}{=}dat\PY{p}{)}\PY{p}{;}\PY{k+kp}{options}\PY{p}{(}repr.plot.width\PY{o}{=}\PY{l+m}{6}\PY{p}{,}repr.plot.height\PY{o}{=}\PY{l+m}{1.5}\PY{p}{,}repr.plot.pointsize\PY{o}{=}\PY{l+m}{7}\PY{p}{)}\PY{p}{;}par\PY{p}{(}mfrow\PY{o}{=}\PY{k+kt}{c}\PY{p}{(}\PY{l+m}{1}\PY{p}{,}\PY{l+m}{2}\PY{p}{)}\PY{p}{)}\PY{p}{;}plot\PY{p}{(}m5\PY{p}{,}which\PY{o}{=}\PY{k+kt}{c}\PY{p}{(}\PY{l+m}{2}\PY{p}{,}\PY{l+m}{3}\PY{p}{)}\PY{p}{)}
\end{Verbatim}


    \begin{center}
    \adjustimage{max size={0.9\linewidth}{0.9\paperheight}}{output_39_0.png}
    \end{center}
    { \hspace*{\fill} \\}
    
    \paragraph{Normality Assumption}\label{normality-assumption}

    We do Shapiro-Wilk normality test to check normality, and got
p-value=0.1039

\(H_{0}\): the residual follows normal distribution.

P-value is larger than \(0.05\), retain \(H_{0}\). So the residual
follows normal distribution.

    \paragraph{Homoscedasticity
Assumption}\label{homoscedasticity-assumption}

    We do Non-constant Variance Score Test to check Homoscedasticity, and
got p value=0.989.

\(H_{0}\): homoscedasticity vs \(H_{1}\):variance residuals vary with
the level of fitted values

P-value is larger than \(0.05\), retain \(H_{0}\). So the residual
follows homoscedasticity assumption.

    \paragraph{Robustness}\label{robustness}

    \begin{longtable}[]{@{}llll@{}}
\toprule
term & Value & Std.Error & t value\tabularnewline
\midrule
\endhead
Intercept & -46.652 & 2.761 & -16.894\tabularnewline
ABDOMEN & 0.922 & 0.051 & 18.255\tabularnewline
WEIGHT:WRIST & -0.013 & 0.002 & -7.558\tabularnewline
\bottomrule
\end{longtable}

    The coefficients of the robust model5 are very close to those of the
previous model5, which means model5 is robust to some extent.

    \subsection{Rules of thumb}\label{rules-of-thumb}

    \[ Bodyfat(\%) = -45 + 0.91\times Abdomen(cm) -0.013\times Weight(kg)\times Wrist(cm)\]

\paragraph{Explain the practical meaning of this
model:}\label{explain-the-practical-meaning-of-this-model}

For a 75 kg man whose abdomen is 85 cm, wrist is 18 cm, his predicted
bodyfat is around 14.97\%. There is a 95\% probability that his bodyfat
is between 14.35\% and 15.59\%. This model is a simpler one to calculate
bodyfat, it predicts this person has 14.8\% bodyfat.

    \subsection{Strengths and weakness}\label{strengths-and-weakness}

    Our model is the most accurate one, compared with other models including
same variables or less variables, and it even has a smaller adjusted
R-squared than the four-variable one. So we think it is a good tradeoff
between simplicity(less variables and less terms) and accuracy(R-squared
and MSE).

Our model can be interpreted in a practical way. It is reasonable that
heavier men's body fat rapidly grows with extremely thicker wrist, while
normal weight males has larger body fat percentage with relatively
thinner wrist.

Our model solves the problem of multicollinearity between independent
variables that some other models have.

Although the model is not complicated, which only has two terms, its
coefficients are hard for people to memorize.

    \paragraph{Contribution and reference}\label{contribution-and-reference}

    {[}1{]} Bodyfat calculator
https://www.active.com/fitness/calculators/bodyfat

{[}2{]} M-estimator https://en.wikipedia.org/wiki/M-estimator


    % Add a bibliography block to the postdoc
    
    
    
    \end{document}
